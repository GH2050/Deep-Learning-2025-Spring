\documentclass[10pt]{beamer}
%================================================
%---------------包配置
%================================================
\usepackage{appendixnumberbeamer}
\usepackage{booktabs}
\usepackage[scale=2]{ccicons}
\usepackage{pgfplots}
\usepackage{xspace}
\newcommand{\themename}{\textbf{\textsc{metropolis}}\xspace}
\renewcommand{\raggedright}{\leftskip=0pt \rightskip=0pt plus 0cm}

% 中文支持 - ShareLaTeX兼容版本
\usepackage[UTF8]{ctex}

\usepackage[T1]{fontenc}
\usepackage[utf8]{inputenc}
\usepackage{fixltx2e}
\usepackage{graphicx}
\usepackage{pdflscape}
\usepackage{epstopdf}
\usepackage{pdfpages}
\usepackage{verbatim}
\tolerance=1000

% ShareLaTeX兼容的代码高亮设置
\usepackage{listings}
\lstset{
    basicstyle=\ttfamily\small,
    breaklines=true,
    frame=single,
    backgroundcolor=\color{gray!10}
}

\usepackage{xcolor}
\usepackage[all]{xy}
\usepackage{times}
\usepackage{tikz}
\usepackage{ragged2e}
\usepackage{caption}
\usepackage{subcaption}
\usepackage{float}

\usepackage{indentfirst}
\usepackage{graphicx}

\usepackage{multirow}
\usepackage{multicol}
\usepackage{lipsum}
\usepackage{ragged2e}
\newcommand{\jus}{\justifying}
\apptocmd{\frame}{}{\justifying}{}
\apptocmd{\block}{}{\justifying}{}
\apptocmd{\column}{}{\justifying}{}
\usepackage{enumerate}

\usepackage{etoolbox}
\usepackage{letltxmacro}

\hypersetup{colorlinks,citecolor=verde,linkcolor=orange,urlcolor=verde}
% 使用标准的natbib代替abntex2cite以确保ShareLaTeX兼容性
\usepackage{natbib}
\bibliographystyle{plainnat}

%=============================
\usetheme[sectionpage=none, subsectionpage=progressbar]{metropolis}
\usepgfplotslibrary{dateplot}
\useoutertheme[subsection=false,footline=authortitle]{miniframes}
\setbeamertemplate{footline}[frame number]
\hypersetup{pdfpagemode=FullScreen}
\hypersetup{pdfpagelayout=SinglePage}

\providecommand{\sin}{} \renewcommand{\sin}{\hspace{2pt}\textrm{sen}}
\providecommand{\tan}{} \renewcommand{\tan}{\hspace{2pt}\textrm{tg}}
\setbeamerfont{section in toc}{size=\small}
\setbeamerfont{subsection in toc}{size=\scriptsize}
\setbeamerfont{subsubsection in toc}{size=\scriptsize}

\AtBeginSection[]{\begin{frame}<beamer>\frametitle{当前章节}\tableofcontents[currentsection]\end{frame}}
\LetLtxMacro{\OldIncludegraphics}{\includegraphics}
\renewcommand{\includegraphics}[2][]{\OldIncludegraphics[height=0.8\textheight,keepaspectratio, #1]{#2}}
\newcommand{\dis}{\displaystyle}
\newcommand{\lb}{\left\lbrace}
\newcommand{\rb}{\right\rbrace}

%================================================
%---------------模板颜色配置
%================================================
\definecolor{verde}{HTML}{008069}
\definecolor{verdepastel}{HTML}{d8f3dc}
\definecolor{verdesolution}{HTML}{beff9e}
\definecolor{vsplbggray}{HTML}{f4f8ff}

\setbeamercolor{alerted text}{fg=orange}
\setbeamercolor*{palette primary}{fg=verde!60!black,bg=gray!30!white}
\setbeamercolor*{palette tertiary}{bg=verde!90,fg=white}
\setbeamercolor*{palette quaternary}{fg=verde,bg=gray!5!white}
\setbeamercolor*{sidebar}{fg=black,bg=black}
\setbeamercolor{progress bar}{fg=verde}
\setbeamercolor*{titlelike}{parent=palette primary}
\setbeamercolor{titlelike}{parent=palette primary,fg=verde}
\setbeamercolor{frametitle}{bg=verdepastel,fg=verde}
\setbeamercolor{frametitle right}{bg=black}
\setbeamercolor{background canvas}{bg=white}

%================================================
%---------------文本块配置
%================================================
\setbeamertemplate{blocks}[rounded][shadow]

\setbeamercolor{block title}{fg=vsplbggray,bg=verde}
\setbeamercolor{block body}{bg=white}
\setbeamercolor{block title alerted}{bg=alerted text.fg!20}
\setbeamercolor{block body alerted}{bg=alerted text.fg!10}

%----------问题块--------------
\definecolor{vblock}{HTML}{E30022}
\BeforeBeginEnvironment{problem}{
\setbeamercolor{block title}{fg=vblock,bg=red!30!white}
\setbeamercolor{block body}{fg=vblock, bg=red!15!white}
}
\AfterEndEnvironment{problem}{
 \setbeamercolor{block title}{use=structure,fg=structure.fg,bg=structure.fg!20!bg}
 \setbeamercolor{block body}{parent=normal text,use=block title,bg=block title.bg!50!bg, fg=black}}

%----------解决方案块--------------
\BeforeBeginEnvironment{solution}{%
\setbeamercolor{block title}{fg=verde,bg=verdesolution}
\setbeamercolor{block body}{fg=verde, bg=green!5!white}
}
\AfterEndEnvironment{solution}{
 \setbeamercolor{block title}{use=structure,fg=structure.fg,bg=structure.fg!20!bg}
 \setbeamercolor{block body}{parent=normal text,use=block title,bg=block title.bg!50!bg, fg=black}
}

%----------示例块--------------
\setbeamercolor{block title example}{use=example text,fg=example text.fg,bg=example text.fg!50!bg}
\setbeamercolor{block body example}{parent=normal text,use=block title example,bg=block title example.bg!50!bg}

%================================================
%---------------封面信息
%================================================
\title{\bfseries 深度学习研究报告}
\subtitle{\textcolor{orange}{先进卷积与注意力机制}}
\date{\textcolor{black}{\today}}
\author{\bf
学生:姓名\\
导师:导师姓名\\
副导师:副导师姓名
}
\institute{\textcolor{verde}{深度学习课程 - 2025年春季学期}}

\begin{document}
\maketitle

%================================================
%---------------目录
%================================================
\begin{frame}{目录}
\hypersetup{linkcolor=black}
\setbeamertemplate{section in toc}[sections numbered]
\tableofcontents
\end{frame}

%================================================
%---------------正文内容
%================================================
\section{引言}

\subsection{研究问题}
\begin{frame}{研究问题}
本研究探讨先进卷积网络和注意力机制在深度学习中的应用。主要关注以下几个方面:
\begin{itemize}
\item 轻量化网络架构设计
\item 注意力机制的有效性
\item 模型性能与计算效率的平衡
\end{itemize}
\end{frame}

\subsection{研究动机}
\begin{frame}{文本块和引用示例}
\begin{block}{如何引用文献?}
\begin{table}[H]
\centering
\begin{tabular}{|l|l|l|}
\hline
引用类型 & 命令 & 结果 \\ \hline
作者引用 & \textbackslash citet\{key\} & 作者等(年份) \\ \hline
括号引用 & \textbackslash citep\{key\} & (作者等,年份) \\ \hline
\end{tabular}
\end{table}
\end{block}

\begin{problem}
传统卷积神经网络存在参数量大、计算复杂度高的问题。
\end{problem}

\begin{solution}
采用GhostNet等轻量化网络架构,结合注意力机制提升模型效率。
\end{solution}
\end{frame}

\subsection{研究目标}
\begin{frame}{研究目标}
本研究的主要目标包括:
\begin{itemize}
\item 实现轻量化卷积网络的训练和优化
\item 分析不同注意力机制的效果
\item 在保持精度的前提下降低计算成本
\end{itemize}

\begin{block}{目标说明}
研究目标将指导我们的实验设计和结果分析,确保研究具有明确的方向性和可评估性。
\end{block}
\end{frame}

\section{文献综述}
\begin{frame}{相关工作}
\begin{itemize}
\item \textbf{轻量化网络}:MobileNet、ShuffleNet、GhostNet等
\item \textbf{注意力机制}:SE-Net、CBAM、ECA-Net等
\item \textbf{网络优化}:知识蒸馏、剪枝、量化等技术
\end{itemize}
\end{frame}

\section{研究方法}
\begin{frame}{实验设置}
\begin{itemize}
\item \textbf{数据集}:CIFAR-10、ImageNet等标准数据集
\item \textbf{模型}:GhostNet、ResNet等网络架构
\item \textbf{训练策略}:数据增强、学习率调度、正则化
\item \textbf{评估指标}:准确率、参数量、FLOPs、推理时间
\end{itemize}
\end{frame}

\section{实验结果}
\begin{frame}{主要结果}
实验结果表明:
\begin{itemize}
\item GhostNet在保持精度的同时显著减少了参数量
\item 注意力机制能够有效提升模型性能
\item 优化策略对模型收敛起到关键作用
\end{itemize}
\end{frame}

\section{总结与展望}
\begin{frame}{结论}
\begin{itemize}
\item 轻量化网络在移动端部署具有重要意义
\item 注意力机制是提升网络性能的有效手段
\item 未来可进一步探索神经架构搜索等自动化方法
\end{itemize}
\end{frame}

%================================================
%---------------结束页面
%================================================
{\setbeamercolor{palette primary}{fg=black, bg=white}
\begin{frame}[standout]
\centering
\Huge{谢谢!}
\vspace{1cm}

\Large{欢迎提问与讨论}
\end{frame}
}

%================================================
%---------------参考文献
%================================================
\section*{参考文献}
\begin{frame}[allowframebreaks]{参考文献}
\begin{thebibliography}{99}
\bibitem{han2020ghostnet}
Han, K., Wang, Y., Tian, Q., et al. (2020). GhostNet: More features from cheap operations. In CVPR.

\bibitem{hu2018squeeze}
Hu, J., Shen, L., Sun, G. (2018). Squeeze-and-excitation networks. In CVPR.

\bibitem{woo2018cbam}
Woo, S., Park, J., Lee, J. Y., Kweon, I. S. (2018). CBAM: Convolutional block attention module. In ECCV.
\end{thebibliography}
\end{frame}

%================================================
%---------------模板结束
%================================================
\end{document}
